%        File: header.tex
%     Created: Mon Aug 26 08:00 PM 2019 P
% Last Change: Mon Aug 26 08:00 PM 2019 P
%
\documentclass[11pt,letterpaper]{article}

\title{Time Delay Device Files Documentation}
\author{Alex Striff \& Lucas Illing}
\date{October 7, 2019}

\begin{document}
\maketitle

All of the documentation concerning the auxilliary files for the project is
assembled here. What follows is a description of the contents of each directory.

\begin{itemize}
  \item \texttt{arduino}: The code uploaded to the Arduino Due.
  \item \texttt{doc}: The top-level documentation directory (this one).
  \item \texttt{fifo-p3f}: The KiCAD files for the device, as well as a
    schematic \textsc{pdf} and the gerber files that I sent for manufacture.
  \item \texttt{figures}: The images and GNUPlot code used to create the figures
    for the paper.\footnote{\emph{Spumpus} stands for spiky lumpus, where
      \emph{lumpus} refers to a lumpy-looking signal. We want a strange signal
    for the programming to demonstrate that it is really \emph{arbitrary}.}
  \item \texttt{program-rescaling}: The code used to calculate the gain and
    offset resistor values to be optimal, given the constraints of E-series
    values and the overall resistances needed.
\end{itemize}

\end{document}



